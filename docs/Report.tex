\documentclass{article}

\usepackage[utf8]{inputenc}
\usepackage[T1]{fontenc}
\usepackage[english]{babel}
\usepackage[pdfborder={0 0 0}]{hyperref}

\author{
    Lukas Prokop,
    Andi
}
\title{Digital library system in flask}
\date{\today}

\begin{document}
\maketitle
\tableofcontents

\section{Technology involved}
%
\begin{description}
  \item[Python]
    Our programming language in charge (version 2.7 is used).
  \item[Flask]
    Flask is a python micro-webframework written by Armin Ronacher.
    It features a very simple and basic API to provide web content easily
    and is extensible by plugins.
  \item[SQLAlchemy]
    SQLAlchemy is used as database binding (object relational mapped)
    to a postgresql database.
  \item[FlaskSQLAlchemy]
    SQLAlchemy binding for Flask.
  \item[PostgreSQL]
    We have decided in favor of postgresql as our database system.
  \item[Jinja2]
    This is our template engine (equivalently its template language).
  \item[Elastic search]
    Provides search capabilities to our dataset.
  \item[PDFMiner]
    A PDF parser capable of extracting the metadata.
\end{description}

\section{How to set it up}
%
\begin{enumerate}
  \item Run \texttt{virtualenv .} and \texttt{. bin/activate}
        as described at \href{http://flask.pocoo.org/docs/installation/#installation}%
        {Flask's installation page}.
  \item Run \texttt{pip install Flask} and \texttt{pip install elasticsearch}
        to install those packages for this repository locally.
  \item Run \texttt{python index.py} in the project's root directory to start the web application.
  \item Visit \url{http://localhost:5000/}.
\end{enumerate}

\section{Database layout}
%
We only use two tables. One table is called \emph{documents} which registers all documents. Attributes are:
\begin{description}
  \item[id] A basic integer primary key.
  \item[type] Which kind of document this is (e.g. ``doc'', ``attach'', ``comment'')
  \item[title] Title of the document
  \item[author] Who created this document?
  \item[timestamp] When was this document created?
  \item[parent] Which is the parent document?
\end{description}

Our second table stores various metadata, which attributes also its name \emph{metadata}.
\begin{description}
  \item[document] Which document is this metadata associated with?
  \item[key] An arbitrary key like \texttt{pdf.author} for the author in the metadata field of the PDF file.
  \item[value] The corresponding value for the given \texttt{key} of \texttt{document}.
\end{description}

\section{How to use it}
%
The requirements state that browsing, searching, inserting and presenting documents must be possible.
This corresponds to the following pages of the web application.
\begin{description}
  \item[Browsing]
    Documents can be browsed in the listing. You have to visit the list page.
  \item[Searching]
    Main (start) page. You can enter various search queries to retrieve documents from the dataset.
    The set of possible search queries is listed at the syntax page.
  \item[Inserting]
    The insertion can be done by visiting the upload page.
  \item[Presenting]
    Each document (and its associated documents) are presented on a separate page.
\end{description}

\end{document}
